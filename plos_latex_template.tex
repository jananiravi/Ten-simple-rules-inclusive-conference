% Template for PLoS
% Version 3.5 March 2018
%
% % % % % % % % % % % % % % % % % % % % % %
%
% -- IMPORTANT NOTE
%
% This template contains comments intended 
% to minimize problems and delays during our production 
% process. Please follow the template instructions
% whenever possible.
%
% % % % % % % % % % % % % % % % % % % % % % % 
%
% Once your paper is accepted for publication, 
% PLEASE REMOVE ALL TRACKED CHANGES in this file 
% and leave only the final text of your manuscript. 
% PLOS recommends the use of latexdiff to track changes during review, as this will help to maintain a clean tex file.
% Visit https://www.ctan.org/pkg/latexdiff?lang=en for info or contact us at latex@plos.org.
%
%
% There are no restrictions on package use within the LaTeX files except that 
% no packages listed in the template may be deleted.
%
% Please do not include colors or graphics in the text.
%
% The manuscript LaTeX source should be contained within a single file (do not use \input, \externaldocument, or similar commands).
%
% % % % % % % % % % % % % % % % % % % % % % %
%
% -- FIGURES AND TABLES
%
% Please include tables/figure captions directly after the paragraph where they are first cited in the text.
%
% DO NOT INCLUDE GRAPHICS IN YOUR MANUSCRIPT
% - Figures should be uploaded separately from your manuscript file. 
% - Figures generated using LaTeX should be extracted and removed from the PDF before submission. 
% - Figures containing multiple panels/subfigures must be combined into one image file before submission.
% For figure citations, please use "Fig" instead of "Figure".
% See http://journals.plos.org/plosone/s/figures for PLOS figure guidelines.
%
% Tables should be cell-based and may not contain:
% - spacing/line breaks within cells to alter layout or alignment
% - do not nest tabular environments (no tabular environments within tabular environments)
% - no graphics or colored text (cell background color/shading OK)
% See http://journals.plos.org/plosone/s/tables for table guidelines.
%
% For tables that exceed the width of the text column, use the adjustwidth environment as illustrated in the example table in text below.
%
% % % % % % % % % % % % % % % % % % % % % % % %
%
% -- EQUATIONS, MATH SYMBOLS, SUBSCRIPTS, AND SUPERSCRIPTS
%
% IMPORTANT
% Below are a few tips to help format your equations and other special characters according to our specifications. For more tips to help reduce the possibility of formatting errors during conversion, please see our LaTeX guidelines at http://journals.plos.org/plosone/s/latex
%
% For inline equations, please be sure to include all portions of an equation in the math environment.  For example, x$^2$ is incorrect; this should be formatted as $x^2$ (or $\mathrm{x}^2$ if the romanized font is desired).
%
% Do not include text that is not math in the math environment. For example, CO2 should be written as CO\textsubscript{2} instead of CO$_2$.
%
% Please add line breaks to long display equations when possible in order to fit size of the column. 
%
% For inline equations, please do not include punctuation (commas, etc) within the math environment unless this is part of the equation.
%
% When adding superscript or subscripts outside of brackets/braces, please group using {}.  For example, change "[U(D,E,\gamma)]^2" to "{[U(D,E,\gamma)]}^2". 
%
% Do not use \cal for caligraphic font.  Instead, use \mathcal{}
%
% % % % % % % % % % % % % % % % % % % % % % % % 
%
% Please contact latex@plos.org with any questions.
%
% % % % % % % % % % % % % % % % % % % % % % % %

\documentclass[10pt,letterpaper]{article}
\usepackage[top=0.85in,left=2.75in,footskip=0.75in]{geometry}

% amsmath and amssymb packages, useful for mathematical formulas and symbols
\usepackage{amsmath,amssymb}

% Use adjustwidth environment to exceed column width (see example table in text)
\usepackage{changepage}

% Use Unicode characters when possible
\usepackage[utf8x]{inputenc}

% textcomp package and marvosym package for additional characters
\usepackage{textcomp,marvosym}

% cite package, to clean up citations in the main text. Do not remove.
\usepackage{cite}

% Use nameref to cite supporting information files (see Supporting Information section for more info)
\usepackage{nameref,hyperref}

% line numbers
\usepackage[right]{lineno}

% ligatures disabled
\usepackage{microtype}
\DisableLigatures[f]{encoding = *, family = * }

% color can be used to apply background shading to table cells only
\usepackage[table]{xcolor}

% array package and thick rules for tables
\usepackage{array}

%strikethrough
\usepackage{soul}

% create "+" rule type for thick vertical lines
\newcolumntype{+}{!{\vrule width 2pt}}

% create \thickcline for thick horizontal lines of variable length
\newlength\savedwidth
\newcommand\thickcline[1]{%
  \noalign{\global\savedwidth\arrayrulewidth\global\arrayrulewidth 2pt}%
  \cline{#1}%
  \noalign{\vskip\arrayrulewidth}%
  \noalign{\global\arrayrulewidth\savedwidth}%
}

% \thickhline command for thick horizontal lines that span the table
\newcommand\thickhline{\noalign{\global\savedwidth\arrayrulewidth\global\arrayrulewidth 2pt}%
\hline
\noalign{\global\arrayrulewidth\savedwidth}}


% Remove comment for double spacing
%\usepackage{setspace} 
%\doublespacing

% Text layout
\raggedright
\setlength{\parindent}{0.5cm}
\textwidth 5.25in 
\textheight 8.75in

% Bold the 'Figure #' in the caption and separate it from the title/caption with a period
% Captions will be left justified
\usepackage[aboveskip=1pt,labelfont=bf,labelsep=period,justification=raggedright,singlelinecheck=off]{caption}
\renewcommand{\figurename}{Fig}

% Use the PLoS provided BiBTeX style
\bibliographystyle{plos2015}

% Remove brackets from numbering in List of References
\makeatletter
\renewcommand{\@biblabel}[1]{\quad#1.}
\makeatother



% Header and Footer with logo
\usepackage{lastpage,fancyhdr,graphicx}
\usepackage{epstopdf}
%\pagestyle{myheadings}
\pagestyle{fancy}
\fancyhf{}
%\setlength{\headheight}{27.023pt}
%\lhead{\includegraphics[width=2.0in]{PLOS-submission.eps}}
\rfoot{\thepage/\pageref{LastPage}}
\renewcommand{\headrulewidth}{0pt}
\renewcommand{\footrule}{\hrule height 2pt \vspace{2mm}}
\fancyheadoffset[L]{2.25in}
\fancyfootoffset[L]{2.25in}
\lfoot{\today}

%% Include all macros below

\newcommand{\lorem}{{\bf LOREM}}
\newcommand{\ipsum}{{\bf IPSUM}}

%% END MACROS SECTION


\begin{document}
\vspace*{0.2in}

% Title must be 250 characters or less.
\begin{flushleft}
{\Large
\textbf\newline{Ten simple rules towards an inclusive conference} % Please use "sentence case" for title and headings (capitalize only the first word in a title (or heading), the first word in a subtitle (or subheading), and any proper nouns).
}
\newline
% Insert author names, affiliations and corresponding author email (do not include titles, positions, or degrees).
\\
Name1 Surname\textsuperscript{1,2\Yinyang},
Name2 Surname\textsuperscript{2\Yinyang},
Name3 Surname\textsuperscript{2,3\textcurrency},
Name4 Surname\textsuperscript{2},
Name5 Surname\textsuperscript{2\ddag},
Name6 Surname\textsuperscript{2\ddag},
Name7 Surname\textsuperscript{1,2,3*},
with the Lorem Ipsum Consortium\textsuperscript{\textpilcrow}
\\
\bigskip
\textbf{1} Affiliation Dept/Program/Center, Institution Name, City, State, Country
\\
\textbf{2} Affiliation Dept/Program/Center, Institution Name, City, State, Country
\\
\textbf{3} Affiliation Dept/Program/Center, Institution Name, City, State, Country
\\
\bigskip

% Insert additional author notes using the symbols described below. Insert symbol callouts after author names as necessary.
% 
% Remove or comment out the author notes below if they aren't used.
%
% Primary Equal Contribution Note
\Yinyang These authors contributed equally to this work.

% Additional Equal Contribution Note
% Also use this double-dagger symbol for special authorship notes, such as senior authorship.
\ddag These authors also contributed equally to this work.

% Current address notes
\textcurrency Current Address: Dept/Program/Center, Institution Name, City, State, Country % change symbol to "\textcurrency a" if more than one current address note
% \textcurrency b Insert second current address 
% \textcurrency c Insert third current address

% Deceased author note
\dag Deceased

% Group/Consortium Author Note
\textpilcrow Membership list can be found in the Acknowledgments section.

% Use the asterisk to denote corresponding authorship and provide email address in note below.
* correspondingauthor@institute.edu

\end{flushleft}
% Please keep the abstract below 300 words
\section*{Abstract (optional from what I've seen)}

In July 2021, the authors of this article participated in the organization team of the annual user conference of the R Project for Statistical Computing. useR! conferences are non-profit events, organized by volunteers from the R community and arranged by the R Foundation. The conference attracts a broad range of participants from academia, industry, government, and the non-profit sector. For 2021, we aimed to build a high-quality virtual and explicitly global conference in a kind, inclusive, accessible, and welcoming environment for everyone. 
In this article, we share a few lessons learned in the process. We streamline our most important learnings in 10 simple rules to host an inclusive conference. These rules apply equally to academic, industry, or mixed conferences; the rules are inspired by a global experience but also apply at the regional or local level.

%andrea: matt's original sentence: reaching users and developers of the R language from more than 120 countries. 

% % Please keep the Author Summary between 150 and 200 words

\linenumbers

\section*{Introduction}

Conferences are spaces to meet and reconnect with members from a specific community, learn about advances in the field, and share our recent contributions. The larger the conference, the larger the opportunities to network and learn from your cohort. However, conferences can become discriminating spaces, in which members of some specific privileged groups (e.g., white, male, from a rich country, English-native speaker, with no physical disabilities) reproduce the systematic inequalities that occur in academia and society \cite{arendDisparityConferenceRegistration2019, timperleyHeMoanaPukepuke2020, gewinWhatScientistsShould2019, brownAbleismAcademiaWhere2018}.

While there has been some action to address systematic inequalities [where], there is still a lot of room for improvement.
% Rocío: Somehow this part below should be linked to the systemic inequalities part
Each one of us can have different points of view and experiences regarding inequality, depending on where we come from.
If you are reading this, you may be here because you already have some experience in Diversity, Inclusion, and Accessibility practices. Or you may want to improve diversity in your conference but do not know how or where to start. This article does not intend to be a comprehensive review of this topic --this would be impossible-- but will give some ideas pointing in key directions.   %Rocío: And finish with something more specific to connect with the next paragraph.

This article suggests rules to pivot traditional conferences towards inclusiveness and diversity. It is directed as people who are part of a stable meetings committee that oversees the site/location selection process or that coordinates with the local organizers (e.g. R Foundation Conference Committee, Ecological Society of America Meetings Committee). It is also directed at the local/virtual organizers who desire to make an inclusive conference starting at the planning stage. 

These tips stem from the authors' experience of organizing useR! 2021, a virtual and global conference for users and developers of the R programming language \cite{r_core_team_2021}. We embraced the challenge of organizing a high-quality virtual conference in the context of the COVID-19 pandemic and making it a kind, truly inclusive and accessible experience for everyone. Most of the authors also have experience of organizing other regional and national academic conferences and events in communities of practice such as R-Ladies and The Carpentries. Here, we share the lessons learned within the past year of organizing this global useR!2021, summarized as 10 simple rules towards an inclusive conference.

\section{Define what diversity and inclusion mean for your conference}
\label{rule_diversity}

Diversity encompasses multiple dimensions: age, physical ability, career stage, gender, gender identity,  geographic origin, language, neurodiversity, race, religion, sexual orientation, and socioeconomic background, to name a few. Human diversity should be celebrated and respected in every way. Nonetheless, there are implicit hierarchies along these axes, and some statuses (e.g., cisgender, white, male, from the US or Europe) hold the privilege of being the default settings for which all systems --including conferences-- are consciously and unconsciously built. As a result, most people in the world fall into some category where they are part of a marginalized group --as opposed to a privileged one. A more diverse and inclusive conference  starts by recognizing that people are diverse, and that some groups face discrimination (some more than others) and might be underrepresented in your community and event \cite{timperleyHeMoanaPukepuke2020} because of these implicit hierarchies. 

Think about the reasons why you want to create a more diverse conference. This may be important to you, whether you come from a privileged background or not. Along this paper, Do not treat diversity as a checklist, do not use people from minoritized groups for image purposes, and do not act as a benefactor or savior but as a genuine collaborator.

%Thinking about diversity and inclusion is thinking about counteracting the structures that hold these hierarchies in place.

Think about the groups that are usually unsupported or discriminated against in your community or event, and about the meaning of "diverse" in your context. Imagine the result you would like to see if you succeeded in organizing a diverse conference according to your vision. Would this translate into an even gender distribution in your speakers? Would it be representation of all races -especially Black people-- among the head organizers, speakers, and attendees? Would it be having LGBTQIA+ friendly-spaces or community participation from key geographic regions? This vision should guide and help hold the organizing team accountable along the way.







\section{Create a safe and welcoming environment}
\label{rule_inclusion}

While it is essential to improve representation towards some of the most visible dimensions of human diversity, such as race, gender, and country of origin, building a truly inclusive environment means taking care of all the other aspects of diversity as well. Having consideration of religious practices, setting specific accommodations for lactating women and child care, having LGBTIQ+-friendly spaces, creating community-only spaces, enforcing the use of pronouns, and treating gender as a non-binary variable are just some examples of decisions that can make inclusion real. 

Paying attention to some dimensions of diversity while neglecting others may have unintended consequences. Lack of representation, unwelcoming --or overtly aggressive-- environments hinder participation (or future participation) of people who could otherwise become active community members. In severe cases, they can divert career paths, affect lives, and exclude people from some fields. %Rocío: it would be nice to have reference here.

Adopting a code of conduct and creating a team to enforce it are key aspects in creating a safe environment during a conference \cite{favaroYourScienceConference2016}. The code of conduct is a document meant to keep the community safe and should state clearly: the unacceptable behaviors, the spaces of the conference in which it applies, the consequences for engaging in unacceptable behavior, and the way to report violations \cite{auroraHowRespondCode2018}. 
The code of conduct should be displayed prominently in several spaces of the conference to deter people from unacceptable behavior. An efficient code of conduct acts as a protection for the community because the people who are the target of unacceptable behavior tend to be the ones with less power or privilege.

The code of conduct team should receive training on how to receive reports, respond to incidents, and communicate their responses. A diverse code of conduct team will be more understanding of power dynamics, and sensitive to discrimination and harassment. We strongly recommend reading How to enforce a Code of Conduct by Valerie Aurora (\url{https://frameshiftconsulting.com/resources/code-of-conduct-book/}) as an excellent starting point for the Code of Conduct team work.


%Some teams gather this vision in a diversity statement (e.g., \url{https://user2021.r-project.org/about/diversity_statement/}) to express the values of the conference and make diversity part of the outreach and communication strategy see Rule \ref{rule_communication}. However, it is important to not stop with a statement alone, but to identify specific goals and tasks that ensure compliance with these broader goals, for example, identifying accessible platforms, developing a detailed code of conduct and a team to enforce it, and preparing accessibility guidelines (see other Rules below).

\section{Have an inclusive and diverse organizing team}
\label{rule_organizing_team}

A genuinely inclusive conference can only be organized by an inclusive and diverse organizing team. Build a team with people from different regions, genders, socioeconomic statuses, and other aspects of diversity. Particular attention should be paid to the usually marginalized groups (see \textbf{Rule \ref{rule_diversity}} discussing the dimensions of diversity lacking in your community and event). To ensure a deeper understanding and smooth communication with different diverse groups, it is essential to create a representative working group that functions as a snapshot of the community at large. If you already have an organizing team, check for biases in its composition. 

%liz: please check if this paragraph connects well with the whole rule. i am using nothing about us without us but also talking about all marginalized groups, in the context of building a team. att: andrea

Gathering a diverse team will only work if there is real inclusion. Disabled people often say: "Nothing about us without us"; the same holds for other dimensions of diversity. This means that the actual life experiences, expertise, and insights from people in marginalized groups are not replaceable with good intentions from people outside these groups. A truly inclusive and welcoming space is one in which everyone in the team is invited and allowed to bring their experience to bear. So, if you plan to organize something related to any community, region, or specific dimension of diversity, involve people from these groups. 

Creating and maintaining such a team and space may seem more challenging than working in homogeneous teams, but the positive outcomes far outweigh any inconveniences. 
Having diverse people in decision-making positions will affect positively all the other aspects of your conference because all the processes will benefit from their input, expertise, and distinct perspectives \cite{hongGroupsDiverseProblem2004}. On the other hand, a diverse team plays an important role at creating a welcoming space because seeing people with similar life experiences occupy public spaces, positions of power, and breaking negative stereotypes (i.e., representation) matters (see Rule \ref{rule_inclusion}). %Representation is an important aspect of diversity and inclusion: seeing people like you, from your country, that speak your language is one of the best way to feel you belong

Most advice when working with a diverse team should apply to every team, but is especially true when working with people from marginalized groups. First, delegate. Don't expect self-nomination and voting to work as mechanisms to counteract systemic inequalities. Nominate directly and offer coordinating positions, even if this means stepping down. Offer mentorship and guidance if you encounter cases of impostor syndrome, doubts about the use of English in communication \ref{rule_language}, or others. Break the expectations about leadership as a lonely task and create smaller, co-led groups, where everyone finds their preferred tasks and gets to take leadership. Splitting the workload and responsibilities should not be done by putting care-taking labors --community building, meeting organization and note taking, conversations with potential partners-- on the hands of women and other minoritized groups, while men take the lead in stereotypical highly-valued tasks (see Rule \ref{rule_spotlight}). 

Words matter. In addition to excluding derogatory or discriminatory language (remember that the code of conduct applies to you too, Rule \ref{rule_inclusion}), make the effort to teach yourself the vocabulary and the best ways to communicate to account for every culture and situation. Do not expect people to teach you --it's not their role-- and accept any feedback without being defensive.

And most importantly, take care of your team! Check on them regularly and make sure everyone is comfortable. This represents a great amount of work and you'll need to support each other in the long run.

%rj: Yani's refs to have just in case:
%https://sites.lsa.umich.edu/scottepage/wp-content/uploads/sites/344/2015/11/pnas.pdf; https://www.pnas.org/content/early/2014/11/13/1407301111; https://www.mckinsey.com/business-functions/organization/our-insights/why-diversity-matters


\section{Have a strong online component of the conference} 
\label{rule_online}

In-person interaction is priceless; however, it is more expensive for some, and even unattainable for others. This inaccessibility is particularly true for global conferences that usually take place in high-income countries, making it financially demanding for international participants and often impossible to attend due to immigration requirements \cite{arendDisparityConferenceRegistration2019,gewinWhatScientistsShould2019}. Online conferences are more inclusive: they do not need a visa or a big budget, and are more accessible to people who may be unable to travel because of health issues or family responsibilities. This means that online conferences have a greater reach, not only in terms of participants but in terms of the tutors and presenters that can participate \cite{atkinsonJournalMedicine20202021}. Furthermore, an online format is more environmentally friendly since it eliminates travel-related emissions \cite{sarabipourChangingScientificMeetings2021,ninerBetterWhomLeveling2021, gattrellComparisonCarbonCosts2021}.

Alternatively, a conference could have a hybrid format with an in-person and online component. This dual format could allow a group of people to interact face-to-face while providing many others the opportunity to participate remotely. The challenge and requirement for this kind of setting would be to make the online component as relevant as the in-person component and not just a consolation prize to the less privileged in the community \cite{ninerBetterWhomLeveling2021}. 



\section{Make the conference accessible to people with disabilities}
\label{rule_accessibility}

Conferences are among the least accessible spaces that people with disabilities may encounter \cite{priceAccessImaginedConstruction2009}.
Ironically, accessibility practices are inclusive not only for people with disabilities but can be beneficial to a broad spectrum of people. For instance, having captions is helpful to deaf and hard-of-hearing people, non-native speakers, and everyone in general. 

If the conference is in person, the venue must be accessible for people with mobility limitations. 
Accommodations such as a quiet space for neurodivergent people should also be provided.
Presenters should always speak into a microphone to make it easier for the hard of hearing and for captioners or interpreters to listen to them. 
Regardless of the conference format --online, in person, or hybrid --images used in communications about the conference should be accompanied by alternative text, while videos should have both captions and sound. 
Platforms for conference registration and abstract submission, websites, and chat platforms --if used --should be screen-reader friendly and keyboard accessible, with low technology requirements (hardware, software, and internet connection). 
Captioning for presentations should be available in more than one language if possible.
These aspects should be tested well in advance of going live.  


Some practices, such as making accessible slides and presentations, are not yet common practice and will require great efforts from the presenters if they are not used to them. 
For that reason, the organizing team should provide accessibility guidelines for slides and presentations, encourage their use, and be available for any questions they may have.  
Among other things, the guidelines should ask for raw and accessible material to the talks before the conference, e.g., R Markdown, HTML, or \TeX{} files; if presentations are pre-recorded, the speakers should include their video and ensure that their face is visible so that deaf and hard-of-hearing people can read their lips if needed (see \url{https://user2021.r-project.org/participation/accessibility/} for example). 

Most importantly, accessibility practices are not afterthoughts that can be dealt with at the last minute. 
They require time and early decision-making \cite{irishIncreasingParticipationUsing2020}. 
Conversely, inaccessible decisions are hard to course-correct, e.g., when finding out too late that a venue is inaccessible for wheelchairs in an in-person conference. 
On the other hand, it is likely that financial constraints may not allow you to make your conference completely accessible and there will be choices to make (e.g. choosing between captions and interpreters). 
Including people that are usually disabled by standard practices in conferences in the organizing team would allow them to take part in the decisions from the beginning, spot right away the inaccessible practices that need to improve (see \textbf{Rule \ref{rule_organizing_team}}), and help make the right choices.



\section{Consciously unbias your spotlight roles}
\label{rule_spotlight}

When choosing or inviting people as keynote speakers, program committee, session chairs and other spotlight roles, it is likely that there will not be much diversity in the first set of names. Many of our biased lists are products of the existing systems that have always privileged some groups of people  \cite{dwyerNoticeWhoScience2021,swartzScienceValueDiversity2019,wongBuildDiversityScience2020,dignazioUnicornsJanitorsNinjas2020}. Rather than deter us, this implicit and systemic bias should encourage us to look further to find great people that are not routinely in the spotlight. 
Ensuring diversity in each of these roles needs to be a deliberate process. We need to go beyond our narrow and often limited networks to look for, reach out to, invite, encourage, and onboard these people until there is ample representation across the diversity spectrum and dimensions. 

Make sure that every selection committee --the committee looking for keynote speakers, the selection committee for abstracts, the prizes and award committees --are also diverse \cite{swartzScienceValueDiversity2019, wongBuildDiversityScience2020}, and ask them to be aware that everyone has an implicit biases, to recognize them, and try to counteract them. An inclusive and diverse organizing team (\textbf{Rule \ref{rule_organizing_team}}) is already a great starting point to overcome this bias in other roles. The regional and local communities in your field are also good sources to tap into. 
For example, for useR! 2021, groups like AfricaR (africa-r.org), R-Ladies (rladies.org), MiR (mircommunity.com), Forwards (forwards.github.io), and LatinR (latin-r.com) were fundamental to reach people for the organizing team, potential presenters, and attendees; some of them even held spaces to help the members of their groups to prepare abstracts for submission. 

Furthermore, consider bringing to the spotlight the people that have contributed to your field in more collaborative  ways \cite{cheng2020x+}. Give awards (see Rule \ref{rule_accessibility}) to those who prepared accessible slides and presentations, for instance, to acknowledge being mindful of inclusiveness. Community building, for instance, is challenging and usually unrewarded when compared to publications or software development \cite{acionWhyChooseCommunity2020}; give the people who deliberately took the time to work on their communities the recognition they deserve. Defy the stereotypical criteria for success by acknowledging these community practices. 
However, do not restrict people from marginalized groups or community-builders to talk or work only in issues related to Diversity, Inclusion, and Accessibility. Recognize their areas of expertise and respect their will regarding participation (or not) in community-building events.


\section{Set fair registration rates and provide financial support when needed}

Conferences, even virtual ones, should have registration fees for two primary reasons. The first one is that preparing the conference involves a lot of concerted effort and costs money (e.g., commercial registration tools, captioning, or conference venue, if in-person) and this translates to the participation fee. The second reason has to do with psychology -- people value more the things they pay for, and there is a lower attendance rate for free events than for events with registration costs \cite{eventbrite_ultimate_2017}. 

On the other hand, if we are aiming for inclusiveness and representation, the socioeconomic context of participants should be taken into account when determining the registration rates. Usually, there is a higher fee for people from the industry than for academia. A lower fee for non-profit organizations, government employees, or freelancers should also be considered. It is important to include discounts for students as well as for postdocs (early career researchers) to encourage their participation \cite{sarabipourChangingScientificMeetings2021, andalibPostdocQueueLabour2018, kaplanPostdocNot2012}. Since postdoc (or trainee) statuses are not always well-defined in academia and can vary for each country, their payment category should be explicitly defined. You can take into account the cost of living in each country using conversion factors (e.g. from the International Comparison Program report of the World Bank \cite{arendDisparityConferenceRegistration2019}). Consider that, while it is traditional for employers to provide --at least some --conference support in the Global North, this is not the case everywhere. 

Offering scholarships or grants to attend the conference may be counterproductive, as they can be seen as more of a competition and attract people that might not need the money (e.g., when their PI could pay for it) but could apply for it to enrich their CV while filtering out people who do not feel entitled to earn them. If the goal is to help the people who really need it, the language needs to make that clear without shrouding it with fancy-sounding phrases or terms that single out the recipients (e.g. `grant recipient' or `diversity scholar', respectively). Do not call it a scholarship or a grant. Ultimately, it is a fee waiver or a discount for the people who need additional financial support. Also, the process for applicants should be simplified. Applying for loans, grants, and scholarships may be an emotionally demanding task. Do not complicate the attendees' lives by asking for long essays to convince a committee that they deserve your support. For this stage of the process, a certain degree of trust goes a long way -- when they say they need the support, it is best to give them the waiver rather than second guess their eligibility. If the conference resources allow for it, you could even take further steps to offer financial support for activities that help them have the time and resources to be at the conference: child care support, transportation fees (if in-person), or internet connection services (if virtual). 

\section{Don't let language restrict high-quality participation}
\label{rule_language}

In international conferences, English is often the official language. Submissions, presentations, tutorials, and workshops are in English. The platforms, the webpage, and official communications are also in English. While English is the primary language in scientific communication and one official language makes it conducive to communicate widely, opening up the conference to other languages could make it less intimidating to people who are not fluent in English \cite{ninerBetterWhomLeveling2021}. Excluding them may potentially lead to missing innovative contributions due to a language barrier. Advertising the conference in several languages and considering having non-English workshops and presentations (with or without captions in English) could help overcome this barrier. For instance, hosting one international day/session per conference might be a great place to start!


\section{Actively reach out to people who have been systematically excluded}
\label{rule_communication}
%Andrea: this rule name should be more explicit towards communication, "promoting"
If part of the community has been historically discriminated against, one should emphasize that they are particularly welcome in this event and that the organization will make it a safe and inclusive environment. Publish the diversity statement (e.g., \url{https://user2021.r-project.org/about/diversity_statement/}), the code of conduct and accessibility guidelines. Fee waivers, financial support, and ease of application/registration should also be advertised; people who fear getting rejected when applying for scholarships and waivers may find it relieving that the process will be supportive rather than discriminatory against underrepresented groups. Advertise the conference in multiple languages (rule \ref{rule_language}). Express this welcoming spirit in your communication strategy (social networks, website, brand, and visual identity) to let people know that they are seen, respected, and welcome; that this is their space and community too. 
% Yani: We need to work a little more here, perhaps with some examples.
% Rocío: could add some ideas from here 
% https://www.science.org.au/files/userfiles/support/emcr/documents/one-page-summary-emcr-improving-diversity-web.pdf

\section{Do not expect the organizing team to work full time and for free}

Having a diverse team and executing inclusive and accessible practices throughout the organizing period --it might be a year or two --may require a lot of effort. The effort is worth it because it strengthens the event and the community making it truly welcoming for everyone. However, having a strong community as the only reward may be enough if you are in a privileged position. Some people and often the minoritized ones do not have institutional support to put time and effort into the organization tasks and do not have the luxury to commit to the organization for free. In addition, tasks such as taking and responding to code of conduct reports, can be emotionally intense work and should be additionally rewarded. Be mindful of each particular context, be flexible with hours or commitments, and revise your budget; if possible, secure funds to offer stipends to them early on. 

\section*{Concluding remarks}

The ten rules stated here can be adapted depending on the conference format and settings.
When organizing useR!2021, we engaged in most of the practices mentioned here, and learned others along the way, so we share them here as part of our learning process. 
We organized useR! during a global pandemic, and as a team, this was a challenging journey. 
Very early on, we tried to recognize that our resources were limited (both in time of the organizing team and financially). 
We had to let go of certain aspects that were important to many of us. 
From that experience, this is our last message: Your conference may not become perfectly inclusive and accessible, but the changes you make will make a difference. 
If more conferences and domains apply these rules, the process will get more streamlined, straightforward, and mainstream to adapt with minimal overhead.
And you will make a change towards healthier, stronger, and more inclusive communities.


% Try not to burn you and the team out trying to make it perfect.  


\section*{Acknowledgments}
The authors of this piece would like to thank every single member of the organizing team of useR! 2021 [ \url{https://user2021.r-project.org/about/global-team/}] for their valuable contribution to an inclusive conference experience, and the R Foundation for charging us with the organization of useR! 2021 and supporting us through the process. 


% \nolinenumbers

% % Either type in your references using
% % \begin{thebibliography}{}
% % \bibitem{}
% % Text
% % \end{thebibliography}
% %
% % or
% %
% % Compile your BiBTeX database using our plos2015.bst
\bibliography{community-science}
% % style file and paste the contents of your .bbl file
% % here. See http://journals.plos.org/plosone/s/latex for 
% % step-by-step instructions.
% % % 
% \begin{thebibliography}{10}

% \bibitem{bib1}
% Conant GC, Wolfe KH.
% \newblock {{T}urning a hobby into a job: how duplicated genes find new
%   functions}.
% \newblock Nat Rev Genet. 2008 Dec;9(12):938--950.

% \bibitem{bib2}
% Ohno S.
% \newblock Evolution by gene duplication.
% \newblock London: George Alien \& Unwin Ltd. Berlin, Heidelberg and New York:
%   Springer-Verlag.; 1970.

% \bibitem{bib3}
% Magwire MM, Bayer F, Webster CL, Cao C, Jiggins FM.
% \newblock {{S}uccessive increases in the resistance of {D}rosophila to viral
%   infection through a transposon insertion followed by a {D}uplication}.
% \newblock PLoS Genet. 2011 Oct;7(10):e1002337.

% \end{thebibliography}



\end{document}

