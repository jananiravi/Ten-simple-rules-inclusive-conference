% Template for PLoS
% Version 3.5 March 2018
%
% % % % % % % % % % % % % % % % % % % % % %
%
% -- IMPORTANT NOTE
%
% This template contains comments intended 
% to minimize problems and delays during our production 
% process. Please follow the template instructions
% whenever possible.
%
% % % % % % % % % % % % % % % % % % % % % % % 
%
% Once your paper is accepted for publication, 
% PLEASE REMOVE ALL TRACKED CHANGES in this file 
% and leave only the final text of your manuscript. 
% PLOS recommends the use of latexdiff to track changes during review, as this will help to maintain a clean tex file.
% Visit https://www.ctan.org/pkg/latexdiff?lang=en for info or contact us at latex@plos.org.
%
%
% There are no restrictions on package use within the LaTeX files except that 
% no packages listed in the template may be deleted.
%
% Please do not include colors or graphics in the text.
%
% The manuscript LaTeX source should be contained within a single file (do not use \input, \externaldocument, or similar commands).
%
% % % % % % % % % % % % % % % % % % % % % % %
%
% -- FIGURES AND TABLES
%
% Please include tables/figure captions directly after the paragraph where they are first cited in the text.
%
% DO NOT INCLUDE GRAPHICS IN YOUR MANUSCRIPT
% - Figures should be uploaded separately from your manuscript file. 
% - Figures generated using LaTeX should be extracted and removed from the PDF before submission. 
% - Figures containing multiple panels/subfigures must be combined into one image file before submission.
% For figure citations, please use "Fig" instead of "Figure".
% See http://journals.plos.org/plosone/s/figures for PLOS figure guidelines.
%
% Tables should be cell-based and may not contain:
% - spacing/line breaks within cells to alter layout or alignment
% - do not nest tabular environments (no tabular environments within tabular environments)
% - no graphics or colored text (cell background color/shading OK)
% See http://journals.plos.org/plosone/s/tables for table guidelines.
%
% For tables that exceed the width of the text column, use the adjustwidth environment as illustrated in the example table in text below.
%
% % % % % % % % % % % % % % % % % % % % % % % %
%
% -- EQUATIONS, MATH SYMBOLS, SUBSCRIPTS, AND SUPERSCRIPTS
%
% IMPORTANT
% Below are a few tips to help format your equations and other special characters according to our specifications. For more tips to help reduce the possibility of formatting errors during conversion, please see our LaTeX guidelines at http://journals.plos.org/plosone/s/latex
%
% For inline equations, please be sure to include all portions of an equation in the math environment.  For example, x$^2$ is incorrect; this should be formatted as $x^2$ (or $\mathrm{x}^2$ if the romanized font is desired).
%
% Do not include text that is not math in the math environment. For example, CO2 should be written as CO\textsubscript{2} instead of CO$_2$.
%
% Please add line breaks to long display equations when possible in order to fit size of the column. 
%
% For inline equations, please do not include punctuation (commas, etc) within the math environment unless this is part of the equation.
%
% When adding superscript or subscripts outside of brackets/braces, please group using {}.  For example, change "[U(D,E,\gamma)]^2" to "{[U(D,E,\gamma)]}^2". 
%
% Do not use \cal for caligraphic font.  Instead, use \mathcal{}
%
% % % % % % % % % % % % % % % % % % % % % % % % 
%
% Please contact latex@plos.org with any questions.
%
% % % % % % % % % % % % % % % % % % % % % % % %

\documentclass[10pt,letterpaper]{article}
\usepackage[top=0.85in,left=2.75in,footskip=0.75in]{geometry}

% amsmath and amssymb packages, useful for mathematical formulas and symbols
\usepackage{amsmath,amssymb}

% Use adjustwidth environment to exceed column width (see example table in text)
\usepackage{changepage}

% Use Unicode characters when possible
\usepackage[utf8x]{inputenc}

% textcomp package and marvosym package for additional characters
\usepackage{textcomp,marvosym}

% cite package, to clean up citations in the main text. Do not remove.
\usepackage{cite}

% Use nameref to cite supporting information files (see Supporting Information section for more info)
\usepackage{nameref,hyperref}

% line numbers
\usepackage[right]{lineno}

% ligatures disabled
\usepackage{microtype}
\DisableLigatures[f]{encoding = *, family = * }

% color can be used to apply background shading to table cells only
\usepackage[table]{xcolor}

% array package and thick rules for tables
\usepackage{array}

%strikethrough
\usepackage{soul}

% create "+" rule type for thick vertical lines
\newcolumntype{+}{!{\vrule width 2pt}}

% create \thickcline for thick horizontal lines of variable length
\newlength\savedwidth
\newcommand\thickcline[1]{%
  \noalign{\global\savedwidth\arrayrulewidth\global\arrayrulewidth 2pt}%
  \cline{#1}%
  \noalign{\vskip\arrayrulewidth}%
  \noalign{\global\arrayrulewidth\savedwidth}%
}

% \thickhline command for thick horizontal lines that span the table
\newcommand\thickhline{\noalign{\global\savedwidth\arrayrulewidth\global\arrayrulewidth 2pt}%
\hline
\noalign{\global\arrayrulewidth\savedwidth}}


% Remove comment for double spacing
%\usepackage{setspace} 
%\doublespacing

% Text layout
\raggedright
\setlength{\parindent}{0.5cm}
\textwidth 5.25in 
\textheight 8.75in

% Bold the 'Figure #' in the caption and separate it from the title/caption with a period
% Captions will be left justified
\usepackage[aboveskip=1pt,labelfont=bf,labelsep=period,justification=raggedright,singlelinecheck=off]{caption}
\renewcommand{\figurename}{Fig}

% Use the PLoS provided BiBTeX style
\bibliographystyle{plos2015}

% Remove brackets from numbering in List of References
\makeatletter
\renewcommand{\@biblabel}[1]{\quad#1.}
\makeatother



% Header and Footer with logo
\usepackage{lastpage,fancyhdr,graphicx}
\usepackage{epstopdf}
%\pagestyle{myheadings}
\pagestyle{fancy}
\fancyhf{}
%\setlength{\headheight}{27.023pt}
%\lhead{\includegraphics[width=2.0in]{PLOS-submission.eps}}
\rfoot{\thepage/\pageref{LastPage}}
\renewcommand{\headrulewidth}{0pt}
\renewcommand{\footrule}{\hrule height 2pt \vspace{2mm}}
\fancyheadoffset[L]{2.25in}
\fancyfootoffset[L]{2.25in}
\lfoot{\today}

%% Include all macros below

\newcommand{\lorem}{{\bf LOREM}}
\newcommand{\ipsum}{{\bf IPSUM}}

%% END MACROS SECTION


\begin{document}
\vspace*{0.2in}

% Title must be 250 characters or less.
\begin{flushleft}
{\Large
\textbf\newline{Ten simple rules towards an inclusive conference} % Please use "sentence case" for title and headings (capitalize only the first word in a title (or heading), the first word in a subtitle (or subheading), and any proper nouns).
}
\newline
% Insert author names, affiliations and corresponding author email (do not include titles, positions, or degrees).
\\
Name1 Surname\textsuperscript{1,2\Yinyang},
Name2 Surname\textsuperscript{2\Yinyang},
Name3 Surname\textsuperscript{2,3\textcurrency},
Name4 Surname\textsuperscript{2},
Name5 Surname\textsuperscript{2\ddag},
Name6 Surname\textsuperscript{2\ddag},
Name7 Surname\textsuperscript{1,2,3*},
with the Lorem Ipsum Consortium\textsuperscript{\textpilcrow}
\\
\bigskip
\textbf{1} Affiliation Dept/Program/Center, Institution Name, City, State, Country
\\
\textbf{2} Affiliation Dept/Program/Center, Institution Name, City, State, Country
\\
\textbf{3} Affiliation Dept/Program/Center, Institution Name, City, State, Country
\\
\bigskip

% Insert additional author notes using the symbols described below. Insert symbol callouts after author names as necessary.
% 
% Remove or comment out the author notes below if they aren't used.
%
% Primary Equal Contribution Note
\Yinyang These authors contributed equally to this work.

% Additional Equal Contribution Note
% Also use this double-dagger symbol for special authorship notes, such as senior authorship.
\ddag These authors also contributed equally to this work.

% Current address notes
\textcurrency Current Address: Dept/Program/Center, Institution Name, City, State, Country % change symbol to "\textcurrency a" if more than one current address note
% \textcurrency b Insert second current address 
% \textcurrency c Insert third current address

% Deceased author note
\dag Deceased

% Group/Consortium Author Note
\textpilcrow Membership list can be found in the Acknowledgments section.

% Use the asterisk to denote corresponding authorship and provide email address in note below.
* correspondingauthor@institute.edu

\end{flushleft}
% Please keep the abstract below 300 words
\section*{Abstract (optional from what I've seen)}

In July 2021, the authors of this article participated in the organization team of the annual user conference of the R Project for Statistical Computing. useR! conferences are non-profit events, organized by volunteers from the R community and arranged by the R Foundation. The conference attracts a broad range of participants from academia, industry, government, and the non-profit sector. For 2021, we aimed to build a high-quality virtual and explicitly global conference in a kind, inclusive, accessible, and welcoming environment for everyone. 
In this article, we share a few lessons learned in the process. We streamline our most important learnings in 10 simple rules to host an inclusive conference. These rules apply equally to academic, industry, or mixed conferences; the rules are inspired by a global experience but also apply at the regional or local level.

%ast: matt's original sentence: reaching users and developers of the R language from more than 120 countries. 

% % Please keep the Author Summary between 150 and 200 words

\linenumbers

\section*{Introduction}

Conferences are spaces to meet and reconnect with members from a specific community, learn about advances in the field, and share our recent contributions. The larger the conference, the larger the opportunities to network and learn from your cohort. However, conferences can become an exclusive space for privileged groups (e.g., white, male, from a rich country, English-native speaker, with no physical disabilities)  \cite{arendDisparityConferenceRegistration2019, timperleyHeMoanaPukepuke2020, gewinWhatScientistsShould2019, brownAbleismAcademiaWhere2018}.

%ast will write a paragraph about why diversity and the systemic inequalities
While there has been some action to address systematic inequalities, there is still a lot of room for improvement.

This article suggests rules to pivot traditional conferences towards inclusiveness and diversity and to welcome minoritized groups. It is directed as people who are part of a stable meetings commitee that oversees the site/location selection process or that coordinates with the local organizers (e.g. R Foundation Conference Committee, Ecological Society of America Meetings Committee). It is also directed at the local/virtual organizers who desire to make an inclusive conference starting at the planning stage. 

These tips stem from the authors' experience of organizing useR! 2021, a virtual and global conference for users and developers of the R programming language \cite{r_core_team_2021}. We embraced the challenge of organizing a high-quality virtual conference in the context of the COVID-19 pandemic and making it a kind, truly inclusive, accessible, and welcoming experience for everyone. Most of the authors also have experience of organizing other regional and national academic conferences and events in communities of practice such as R-Ladies and The Carpentries. Here, we share the lessons learned within the past year of organizing this global useR! conference, summarized as 10 simple rules towards an inclusive conference.

\section{Rule X: Define what diversity and inclusion mean for your conference}
\label{rule_diversity}


The first step towards a diverse and inclusive conference is to recognize that people are diverse, that some groups face discrimination and might be underrepresented in your community and event \cite{timperleyHeMoanaPukepuke2020}. 

When thinking about diversity, each one of us can have different points of view about the meaning of the term, depending on where we come from. If you are reading this, you may be here because you already have 
Diversity encompasses multiple dimensions: age, physical ability, career stage, gender, gender identity,  geographic origin, language, neurodiversity, race, religion, sexual orientation, and socioeconomic background, to name a few. While human diversity should be celebrated and respected in every space, the truth is that there are implicit hierarchies along these axes, and some statuses (e.g., cisgender, white, male, from the US or Europe) hold the privilege of being the default settings for which all systems are consciously and unconsciously built. Thinking about diversity and inclusion is thinking about counteracting the structures that hold these hierarchies in place.


Think about the groups that are usually unsupported or discriminated against in your community or event, and about the meaning of "diverse" in your context. Imagine the result you would like to see if you succeeded in organizing a diverse conference according to your definition. Does this mean an even gender distribution in your speakers? Does this relate to people of all races -especially black people- being present as coordinators, presenters, and in the audience? Does this mean having LGBTQ+ friendly-spaces or community participation from key geographic regions? This vision should guide and help hold the organizing team accountable along the way.

To accomplish this diversity vision, you will have to take some conscious decisions toward removing bias from your conference's spaces. Having diverse people in decision-making positions will affect positively all the other aspects of your conference, not only internally because all the decision-making process will benefit from their input (see Rule \ref{rule_organizing_team}), but also externally because representation matters[see Rule \ref{rule_communication}].

While you may want to improve representation towards some of the most visible dimensions of human diversity, such as race, gender, and country of origin, building a truly inclusive environment means taking care of all the other aspects as well. Having consideration of religious practices, setting specific accommodations for lactating women and child care, having LGBTIQ+-friendly spaces, creating community-only spaces, enforcing the use of pronouns, and treating gender as a non-binary variable, and accomodating are some examples of decisions that can make inclusion real. 
%Representation is an important aspect of diversity and inclusion: seeing people like you, from your country, that speak your language is one of the best way to feel you belong
Some teams gather this vision in a diversity statement (e.g., \url{https://user2021.r-project.org/about/diversity_statement/}) to express the values of the conference and make diversity part of the communication strategy see Rule \ref{rule_communication} and in every decision you make, will let people know that they are seen, respected, and welcome; that this is their space and community too. 
However, it is important to not stop with a statement alone, but to identify specific goals and tasks that ensure compliance with these larger goals, for example, identifying accessible platforms, developing a detailed code of conduct and a team to enforce it, and preparing accessibility guidelines (see other Rules below).

Lack of representation, unwelcoming -or overtly aggressive- environments hinder participation (or future participation) of people who could otherwise become active community members. In extreme cases, they can divert career paths, affect lives, and exclude people from some fields. So take this seriously, think about the reasons why you want to create a more diverse conference, and be open to learn. Do not treat diversity as a checklist, do not use people from minoritized groups for image purposes (tokenization), and do not act as a savior but as a genuine collaborator. %ast: some version of this might come in a box.




\section{Rule X: Have a strong online component of the conference} 
\label{rule_online}
In-person interaction is priceless; however, it is more expensive for some, even unattainable for others. This inaccessibility is particularly true for global conferences that usually take place in high-income countries, making it financially demanding for international participants and often impossible to attend due to immigration requirements \cite{arendDisparityConferenceRegistration2019,gewinWhatScientistsShould2019}. Online conferences are more inclusive: they do not need a visa or a big budget, and are more accessible to people who may be unable to travel because of health issues or family responsibilities. This means that online conferences have a greater reach, not only in terms of participants but in terms of the tutors and presenters that can participate \cite{atkinsonJournalMedicine20202021}. Furthermore, an online format is more environmentally friendly since it eliminates travel-related emissions \cite{sarabipourChangingScientificMeetings2021,ninerBetterWhomLeveling2021, gattrellComparisonCarbonCosts}.

Alternatively, a conference could have a hybrid format with an in-person and online component. This dual format could allow a group of people to interact face-to-face while providing many others the opportunity to participate remotely. The challenge in this kind of setting would be to make the online component as relevant as the in-person component and not just a consolation prize to the less privileged in the community \cite{ninerBetterWhomLeveling2021}. 


\section{Rule X: Have an inclusive and diverse organizing team}
\label{rule_organizing_team}
A genuinely inclusive conference can only be organized by an inclusive and diverse organizing team. Build a team with people from different regions, genders, socioeconomic statuses, and other aspects of diversity. Particular attention should be paid to the usually marginalized groups (see \textbf{Rule 1} discussing the dimensions of diversity lacking in your community and event). To ensure a deeper understanding of the challenges in different diverse groups, it is essential to create a representative working group that functions as a snapshot of the community at large. But this will only work if there is real inclusion. One would be surprised by how often conference leads tend to do everything themselves instead of reaching out to the right people for the right task -- likely because the latter involves careful planning and communication well ahead of time or because they naively overestimate their expertise in areas they lack practical experience in. For example, a cisgender or abled person estimating what might work best for transgender people or people with sensory disabilities. The same holds for other dimensions of diversity. A truly inclusive and welcoming space is one in which everyone in the team is invited and allowed to bring their experience to bear. Even if creating and maintaining such a team and space is challenging, the positive outcomes far outweigh any minor inconveniences (e.g. time zones), and the team will grow, learn, and embrace more inclusive practices \cite{hongGroupsDiverseProblem2004}. 

%The team may also have to review carefully and teach themselves the correct vocabulary for internal and external communication and identify the best ways to account for every culture and situation. % rj: left this out, Heather said it doesn't make much sense here %ast: we can collect part of this in rule 1, since some self-teaching, and not expecting to be taught, is necessary

%%Disabled people often repeat "Nothing about us without us", and this is true for every marginalized group. This means the actual life experiences, expertise, and insights from people in marginalized groups are not replaceable with good intentions from people outside these groups. 
%rj: Yani's refs to have just in case:
%https://sites.lsa.umich.edu/scottepage/wp-content/uploads/sites/344/2015/11/pnas.pdf; https://www.pnas.org/content/early/2014/11/13/1407301111; https://www.mckinsey.com/business-functions/organization/our-insights/why-diversity-matters

\section{Rule X: Make the conference accessible to people with disabilities}
\label{rule_accessibility}

Conferences are among the least accessible spaces that people with disabilities may encounter \cite{priceAccessImaginedConstruction2009}.
%People may also find conferences inaccessible due to their caring responsibilities, dietary restrictions, religious practices, or LGBTQ+ status.
Ironically, accessibility practices are inclusive not only for people with disabilities but can be beneficial to a broad spectrum of people. For instance, having captions is helpful to deaf and hard-of-hearing people, non-native speakers, and everyone in general. 

If the conference is in person, the venue must be accessible for people with mobility limitations. 
Accommodations such as a quiet space for neurodivergent people should also be provided.
Presenters should always speak into a microphone to make it easier for the hard of hearing and for captioners or interpreters to listen to them. 
Regardless of the conference format --online, in person, or hybrid --images used in communications about the conference should be accompanied by alternative text, while videos should have both captions and sound. 
Platforms for conference registration and abstract submission, websites, and chat platforms --if used --should be screen-reader friendly and keyboard accessible, with low technology requirements (hardware, software, and internet connection). 
Captioning for presentations should be available in more than one language if possible.
These aspects should be tested well in advance of going live.  

Some practices, such as making accessible slides and presentations, are not yet common practice and will require great efforts from the presenters if they are not used to them. 
For that reason, the organizing team should provide accessibility guidelines for slides and presentations, encourage their use, and be available for any questions they may have.  
Among other things, the guidelines should ask for raw and accessible material to the talks before the conference, e.g., R Markdown, HTML, or \TeX{} files; if presentations are pre-recorded, the speakers should include their video and ensure that their face is visible so that deaf and hard-of-hearing people can read their lips if needed (see \url{https://user2021.r-project.org/participation/accessibility/} for example). 
Accessibility awards at the end of the conference are an excellent way to acknowledge the speakers who were mindful of inclusiveness when preparing and delivering their talks. 

Most importantly, accessibility practices are not afterthoughts that can be dealt with at the last minute. 
They require time and early decision-making \cite{irishIncreasingParticipationUsing2020}. 
Conversely, inaccessible decisions are hard to course-correct, e.g., when finding out too late that a venue is inaccessible for wheelchairs in an in-person conference. 
Welcoming people that are usually disabled by standard practices in conferences into the organizing team would allow them to take part in the decisions from the beginning and spot right away the inaccessible practices that need to improve (see \textbf{Rule \ref{rule_organizing_team}}). 


\section{Rule X: Adopt a code of conduct and create a team to reinforce it}
\label{rule_CoC}

To ensure an inclusive, friendly, and safe space in the conference, the organizers need to adopt a code of conduct and set up a team to enforce it \cite{favaroYourScienceConference2016}. The code of conduct is a document meant to keep the community safe and should state clearly the unacceptable behaviors, the spaces of the conference in which it applies, the consequences for engaging in unacceptable behavior, and the way to report violations \cite{auroraHowRespondCode2018}. 
The code of conduct should be displayed prominently in several spaces of the conference to deter people from unacceptable behavior.

The code of conduct team should receive training on how to receive reports, respond to incidents, and communicate their responses. A diverse code of conduct team will be more understanding of intersectionality issues in discrimination and harassment practices. 
%The people who need the protection of a code of conduct are usually those with less power or privilege, as more powerful or privileged people are often already protected from most harm.'

\section{Rule X: Search proactively for the best people the conference's vision}
\label{rule_presenters}
When inviting people as keynote speakers, program committee, session chairs and others, we should always look for the best in their subject. Ensuring diversity in each of these roles is a deliberate process -- if there is not much diversity in the first set of names, it needs to be revised to include people from minoritized groups. If these groups are not showing up in the A-list, it is seldom because of their technical acumen -- it is because we, as organizers, are not used to looking in unfamiliar places to bring in the talent. Many of our biased lists are products of the existing systems that have always privileged some groups of people \cite{dwyerNoticeWhoScience2021,sarabipourChangingScientificMeetings2021}. Rather than deter us, this should encourage us to look further to find great people that are not routinely in the spotlight. We need to go beyond our narrow and often limited networks to look for, reach out to, invite, encourage, and onboard these people until there is ample representation across the diversity spectrum and dimensions. 

% So, as long as we start with an inclusive and diverse organizing team (\textbf{Rule 2}), we can aim at finding the right people for *these invited roles*. The regional and local communities are also good sources to tap into. For example, for useR! 2021, groups like AfricaR (africa-r.org), R-Ladies (rladies.org), MiR (mircommunity.com), Forwards (forwards.github.io), and LatinR (latin-r.com) were fundamental to reach people for the organizing team, potential attendees, and sponsors. %ast: i think this could come back, removing the specific part about "conference organization" - i edited that. 



\section{Rule X: Set fair registration rates and provide financial support when needed}

Conferences, even virtual ones, should have registration fees for two primary reasons. The first one is that preparing the conference involves a lot of concerted effort and costs money (e.g., commercial registration tools, captioning, or conference venue, if in-person) and this translates to the participation fee. The second reason has to do with psychology -- people value more the things they pay for, and there is a lower attendance rate for free events than for events with registration costs \cite{eventbrite_ultimate_2017}. 

On the other hand, if we are aiming for inclusiveness and representation, the socioeconomic context of participants should be taken into account when determining the registration rates. Usually, there is a higher fee for people from the industry than for academia. A lower fee for non-profit organizations, government employees, or freelancers should also be considered. It is important to include discounts for students as well as for postdocs (early career researchers) to encourage their participation \cite{sarabipourChangingScientificMeetings2021, andalibPostdocQueueLabour2018, kaplanPostdocNot2012}. Since postdoc (or trainee) statuses are not always well-defined in academia and can vary for each country, their payment category should be explicitly defined. You can take into account the cost of living in each country using conversion factors (e.g. from the International Comparison Program report of the World Bank \cite{arendDisparityConferenceRegistration2019}). Consider that, while it is traditional for employers to provide --at least some --conference support in the Global North, this is not the case everywhere. 



%rationale: 1. they might select for the already privileged or people who dont need those to be nice on the cvs 2. they select for outgoing extroverted privileged ("brave") people 3. they are time and emotionally consuming which selects out some people 4. they single out people. even if you controlled for 1 to 3 you would "badge" people - you are creating a minority while trying to avoid it.
Offering scholarships or grants to attend the conference may be counterproductive, as they can be seen as more of a competition and attract people that might not need the money (e.g., when their PI could pay for it) but could apply for it to enrich their CV while filtering out people who do not feel entitled to earn them. If the goal is to help the people who really need it, the language needs to make that clear without shrouding it with fancy-sounding phrases or terms that single out the recipients (e.g. `grant recipient' or `diversity scholar', respectively). Do not call it a scholarship or a grant. Ultimately, it is a fee waiver or a discount for the people who need additional financial support. Also, the process for applicants should be simplified. Applying for loans, grants, and scholarships may be an emotionally demanding task. Do not complicate the attendees' lives by asking for long essays to convince a committee that they deserve your support. For this stage of the process, a certain degree of trust goes a long way -- when they say they need the support, it is best to give them the waiver rather than second guess their eligibility. If the conference resources allow for it, you could even take further steps to offer financial support for activities that help them have the time and resources to be at the conference: child care support, transportation fees (if in-person), or internet connection services (if virtual). 

\section{Rule X: Don't let language restrict high-quality participation}
\label{rule_language}

In international conferences, English is often the official language. Submissions, presentations, tutorials, and workshops are in English. The platforms, the webpage, and official communications are also in English. While English is the primary language in scientific communication and one official language makes it conducive to communicate widely, opening up the conference to other languages could make it less intimidating to people who are not fluent in English \cite{ninerBetterWhomLeveling2021}. Excluding them may potentially lead to missing innovative contributions due to a language barrier. Advertising the conference in several languages and considering having non-English workshops and presentations (with or without captions in English) could help overcome this barrier. For instance, hosting one international day/session per conference might be a great place to start!


\section{Rule X: Actively reach out and involve people who have been systematically excluded}
\label{rule_communication}
%this rule name should be more explicit towards communication, "promoting"
If part of the community has been historically discriminated against, one should emphasize that they are particularly welcome in this event and that the organization will make it a safe and inclusive environment. Publish a diversity statement (e.g., \url{https://user2021.r-project.org/about/diversity_statement/}), the code of conduct and accessibility guidelines. Fee waivers, financial support, and ease of application/registration should also be advertised; people who are used to applying and getting rejected for scholarships and waivers may find it relieving that the process will be supportive rather than discriminatory against underrepresented groups. Advertise the conference in multiple languages (rule \ref{rule_language}). Express this welcoming spirit in your communication strategy (social networks, website, brand, and visual identity) to let people know that they are seen, respected, and welcome; that this is their space and community too. 



\section{Rule X: Do not expect the organizing team to work full time and for free}

Having a diverse team and executing inclusive and accessible practices throughout the organizing period --it might be a year or two --may require a lot of effort. The effort is worth it because it strengthens the event and the community making it truly welcoming for everyone. However, having a strong community as the only reward may be enough if you are in a privileged position. Some people and often the minoritized ones do not have institutional support to put time and effort into the organization tasks and do not have the luxury to commit to the organization for free. In addition, tasks such as taking and responding to code of conduct reports, can be emotionally intense work and should be additionally rewarded. Be mindful of each particular context, be flexible with hours or commitments, and revise your budget; if possible, secure funds to offer stipends to them early on. 


\section{Rule X: Don't let perfect be the enemy of good}
 While conference teams should have a bold vision and clear targets consensus about where to shine and where to compromise is equally important. As a team, this is a challenging journey. It means letting go of some ideals, or committing time and money into something that one does not find particularly useful or important.

Remember, aspirations can be local and narrow and as a consequence shy away valuable contributors. A late bus is almost certainly better than no bus. A broader pool of contributors does not only allow organizers to add more and work on multiple aspects at a different pace, it also helps to split workload and responsibilities to avoid burning out over the marathon that organizing a conference is.

Without allowing for compromises, workload will naturally channel to those who are committed the most and already carry the largest share of the load. Hence compromises are no sign of giving in but are crucial to delegation, building self-responsible teams and keep a good spirit in the long run.   






\section*{Concluding remarks}

The ten rules stated here can be adapted depending on the conference format and settings. From our own experience, %it's worth it :P
we are aware that enforcing these rules requires a tremendous amount of work. We engaged in most of these practices, and learned others too late, so we share them here as part of our learning process. 
Since accessibility and inclusiveness are not as widespread as they should be, the process might seem exacting. But this is where the power of communities and individuals reside. If more conferences and domains employ these techniques, the process will get more streamlined, straightforward, and mainstream to adapt with minimal overhead. And we will make a change towards healthier, stronger, and more inclusive communities, empowering historically marginalized people and learning from them, thereby increasing the overall quality of our conferences.%After all, the strength of any community lies in its ability to be adaptive, inclusive, and accessible.



\section*{Acknowledgments}
The authors of this piece would like to thank every single member of the organizing team of useR! 2021 [ \url{https://user2021.r-project.org/about/global-team/}] for their valuable contribution to an inclusive conference experience, and the R Foundation for charging us with the organization of useR! 2021 and supporting us through the process. 


% \nolinenumbers

% % Either type in your references using
% % \begin{thebibliography}{}
% % \bibitem{}
% % Text
% % \end{thebibliography}
% %
% % or
% %
% % Compile your BiBTeX database using our plos2015.bst
\bibliography{community-science}
% % style file and paste the contents of your .bbl file
% % here. See http://journals.plos.org/plosone/s/latex for 
% % step-by-step instructions.
% % % 
% \begin{thebibliography}{10}

% \bibitem{bib1}
% Conant GC, Wolfe KH.
% \newblock {{T}urning a hobby into a job: how duplicated genes find new
%   functions}.
% \newblock Nat Rev Genet. 2008 Dec;9(12):938--950.

% \bibitem{bib2}
% Ohno S.
% \newblock Evolution by gene duplication.
% \newblock London: George Alien \& Unwin Ltd. Berlin, Heidelberg and New York:
%   Springer-Verlag.; 1970.

% \bibitem{bib3}
% Magwire MM, Bayer F, Webster CL, Cao C, Jiggins FM.
% \newblock {{S}uccessive increases in the resistance of {D}rosophila to viral
%   infection through a transposon insertion followed by a {D}uplication}.
% \newblock PLoS Genet. 2011 Oct;7(10):e1002337.

% \end{thebibliography}



\end{document}

